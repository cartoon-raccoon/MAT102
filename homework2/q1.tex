\documentclass[12pt, a4paper]{article}

\usepackage{amsmath}
\usepackage{amssymb}
\usepackage{amsthm}
\usepackage[a4paper, portrait, margin=1in]{geometry}

\newcommand{\R}{\mathbb{R}}
\newcommand{\open}[2]{\left(#1, #2\right)}

\renewcommand{\qedsymbol}{$\blacksquare$}

\newtheorem{theorem}{Theorem}

% let's begin
\begin{document}

\noindent\textbf{(Q1)}

\noindent (a)

$D = \R \setminus \{-2\}$

\noindent (b)

The range of $f(x)$ is $\open{-\infty}{3} \cup \open{3}{\infty} = \R \setminus \{3\}$.

\begin{proof}
    Let
    \[
        y = f(x) = \frac{3x - 1}{x + 2},\: x \in D,\:y \in \R
    \]

    We can derive a function from $f$ where the range of $f$ becomes the domain 
    of the function, by making $x$ the subject of the equation.

    Thus,
    \begin{align*}
        y = f(x) & = \frac{3x - 1}{x + 2}\\
        \implies yx - 2y & = 3x + 1\\
        \implies yx - 3x & = 2y + 1\\
        \implies x(y - 3) & = 2y + 1\\
        \implies x & = \frac{2y + 1}{y - 3}\\
    \end{align*}

    From this we can deduce that $y$ is defined for all $\R$ except $y = 3$, which is the
    set $\R \setminus \{3\}$.
\end{proof}

\end{document}