\documentclass[12pt, a4paper]{article}

\usepackage{amsmath}
\usepackage{amssymb}
\usepackage{amsthm}
\usepackage[a4paper, portrait, margin=1in]{geometry}

\newcommand{\R}{\mathbb{R}}
\newcommand{\Z}{\mathbb{Z}}
\newcommand{\N}{\mathbb{N}}
\renewcommand{\qedsymbol}{$\blacksquare$}

\newcommand{\sse}{\subseteq}

\newcommand{\emptyline}{\hfill\break}

\newtheorem{theorem}{Theorem}

% let's begin
\begin{document}

\noindent\textbf{(Q2)}

\noindent (a) $2(x)(x + 1)(x + 2)$

\emptyline
\noindent (b) Since $\Z$ is closed under addition and multiplication, we can 
express any integer $a$ in the form
\[
    a = dq - b, 0 \leq b < d
\]
Where $d$, $q$, and $b$ are integers. This shows that any integer can be expressed
as the product of a divisor and quotient, minus a positive remainder $b$ that is always
less than $d$. $a$ is said to be divisible by $d$ iff $b = 0$.

Where the divisor is 3, it follows that
\begin{align*}
    a & = 3q - b, 0 \leq b < 3\\
    \implies a + b & = 3q
\end{align*}

For any three consecutive integers $a + 0$, $a + 1$, and $a + 2$, the constraints on the 
value of $b$ are always satisfied, and there is always an $a$ where $b$ is zero, 
therefore exactly one integer out of any three consecutive integers is divisible by 3.

\emptyline
\noindent(c)

\begin{proof}
    From (b), given three consecutive integers, one is divisible by 3. Let $x \in \N$.
    
    Therefore, given
    a set $\{x, x + 1, x + 2\},\: x \in \N$, taking $x$ to be the integer divisible by 3, we can express
    the set as $\{3q, x + 1, x + 2\}$, where $q \in \N$.

    From (a), we can see that the numerator of the rational function $\tfrac{2x^3 + 6x^2 + 4x}{3}$
    can be factorized to $2(x)(x + 1)(x + 2)$, such that
    \begin{align*}
        f(x) & = \frac{2(x)(x + 1)(x + 2)}{3} \\
        & = \frac{2(3q)(x + 1)(x + 2)}{3} \quad\text{[from (b)]}\\
        & = 2(q)(x + 1)(x + 2)
    \end{align*}
    Since $q, x \in \N$ and $\N$ is closed under addition and multiplication, we can conclude
    that $f(x) \in \N$ for all $x \in \N$.

\end{proof}

\emptyline
\noindent(d)
$f(\N) \neq \N$

\begin{proof}
    Since $f(\N) = \N$ iff $f(\N) \sse \N$ and $\N \sse f(\N)$, to prove 
    that $f(\N) \neq \N$ it suffices to show $\N \nsubseteq f(\N)$.

    Looking at the function, we have $2(q)(x + 1)(x + 2),\: x, q \in \N$.

    From the definition of the function, we can see that $2$ is a factor, thus
    the output of $f(x)$ for all $x \in \N$ will be an even number.

    Since the set of odd numbers is not in the image of $f$, we can conclude that
    $\N \nsubseteq f(\N)$ and thus $f(\N) \neq \N$.
\end{proof}
\end{document}