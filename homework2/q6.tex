\documentclass[12pt, a4paper]{article}

\usepackage{amsmath}
\usepackage{amssymb}
\usepackage{array}
\usepackage{xcolor}
\usepackage[a4paper, portrait, margin=1in]{geometry}

\newcommand{\R}{\mathbb{R}}
\newcommand{\Z}{\mathbb{Z}}
\newcommand{\F}{\mathbb{F}}

\newcommand{\red}[1]{\textcolor{red}{#1}}

% let's begin
\begin{document}

\noindent\textbf{(Q5)}

\noindent(a)

These are the addition and multiplication tables for the given field $\F$.

Entries in \red{red} are entries that were pre-populated by the question.

The addition table is as follows: \hfill\break

\begin{tabular}{ |m{1cm} | m{1cm} | m{1cm} | m{1cm}|}
    \hline
    + & 0 & 1 & $x$\\
    \hline
    0 & 0 & 1 & $x$\\
    \hline
    1 & 1 & \red{$x$} & 0\\
    \hline
    $x$ & $x$ & 0 & \red{1}\\
    \hline
\end{tabular}
\hfill\break

The first row and first column were filled in as such to ensure Axiom 3 was 
fulfilled, specifically the existence of the additive identity. The remaining
zeroes were  filled in as such to ensure that Axiom 4 was fulfilled
(existence of negatives).

The multiplication table is as follows: \hfill\break

\begin{tabular}{ |m{1cm} | m{1cm} | m{1cm} | m{1cm}|}
    \hline
    $\cdot$ & 0 & 1 & $x$\\
    \hline
    0 & 0 & 0 & 0\\
    \hline
    1 & 0 & 1 & \red{$x$}\\
    \hline
    $x$ & 0 & \red{$x$} & 1\\
    \hline
\end{tabular}
\hfill\break

The first row and first column were filled in as such by proof of Claim 2.3.2.

The remaining two ones were filled in as such to ensure Axiom 4 was fulfilled
(existence of reciprocals).

(b)
Following the addition and multiplication tables:
\begin{align*}
    x^2 = x \cdot x & = 1\\
    1 + x & = 0\\
    0 + 1 & = 1
\end{align*}

\end{document}