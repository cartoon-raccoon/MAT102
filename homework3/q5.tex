\documentclass[12pt, a4paper]{article}

\usepackage{amsmath}
\usepackage{amssymb}
\usepackage{amsthm}
\usepackage[a4paper, portrait, margin=1in]{geometry}

\newcommand{\R}{\mathbb{R}}
\newcommand{\Z}{\mathbb{Z}}
\newcommand{\N}{\mathbb{N}}
\renewcommand{\qedsymbol}{$\blacksquare$}

\newcommand{\emptyline}{\hfill\break}

\newtheorem{theorem}{Theorem}

% let's begin
\begin{document}

\noindent\textbf{(Q5)}

All odd numbers $\geq 3$.

We are given that for a statement $P(k)$:

\begin{enumerate}
    \item $P(3)$ is true.
    \item For any $k > 1$, $P(k) \implies P(k + 2)$ is true.
\end{enumerate}

We can justify this using the principle of mathematical induction (PMI). Given a base case
and an induction hypothesis, we can find all numbers $n$ for which $P(n)$ is true.

By (1), since $P(3)$ is true but not $P(1)$ (by (2)), we can take $P(3)$ as our base case.

Statement 2 tells us that $P(k) \implies P(k + 2)$. This means for any given $k \geq 3$,
e.g. 3, $P(k) \implies P(k + 2$) (in this case $P(3) \implies P(5)$). By induction, we can continue this indefinitely
for all odd natural numbers ($P(5) \implies P(7), \: P(7) \implies P(9),$ etc.).

Thus, by PMI, we can conclude that $P(n)$ is true for all odd numbers $n \geq 3$.

\end{document}