\documentclass[12pt, a4paper]{article}

\usepackage{amsmath}
\usepackage{amssymb}
\usepackage{amsthm}
\usepackage[a4paper, portrait, margin=1in]{geometry}

\newcommand{\R}{\mathbb{R}}
\newcommand{\Z}{\mathbb{Z}}
\newcommand{\Q}{\mathbb{Q}}
\newcommand{\N}{\mathbb{N}}
\renewcommand{\qedsymbol}{$\blacksquare$}

\newcommand{\emptyline}{\hfill\break}

\newtheorem{theorem}{Theorem}
\newtheorem{lemma}{Lemma}

% let's begin
\begin{document}

\noindent\textbf{(Q4)}

\begin{theorem}
    $\R \setminus \Q$ is uncountable.
\end{theorem}

This proof requires the following lemma:
\begin{lemma}
    The union of two disjoint countable sets is countable.
\end{lemma}

\begin{proof}
    As always, we prove that a set is countable by forming a bijection between
    $\N$ and the set.

    For the sake of this proof, let the two sets be set $A$ and $B$. $A$ and $B$
    are countable, and thus a bijection exists between $\N$ and $A$ and $B$ respectively.

    Let the bijection for $A$ be given by $f\colon \N \to A$ and the bijection for $B$
    be given by $g \colon \N \to B$.

    We define a function $h \colon \N \to A \cup B$:

    \[
        h(k) = \begin{cases}
            f(\frac{k + 1}{2}) \text{ if $k$ is odd}\\
            g(\frac{k}{2}) \text{ if $k$ is even}
        \end{cases}
    \]

    Thus, the sequence defined by $h(1), h(2), h(3) \ldots$ for $h(\N)$ is:
    \[
        f(1), g(1), f(2), g(2), f(3), g(3) \ldots
    \]

    Since the sets are disjoint, this function is injective, as no two different inputs to $h$ 
    can produce the same outputs. This function is also surjective, since $f$ and $g$ are 
    surjective and map to every element in $A$ and $B$ respectively.

    Thus, $h \colon \N \to A \cup B$ is bijective, and $A \cup B$ is countable.
\end{proof}

We can now prove \textbf{Theorem 1}.

\begin{proof}
    For the sake of contradiction, suppose $\R \setminus \Q$ is countable.

    By earlier proof, $\Q$ is countable. By proof of \textbf{Lemma 1}, the union of
    two disjoint countable sets is also countable. Thus, $(\R \setminus \Q) \;\cup\; \Q$ is countable. 
    However, $(\R \setminus \Q) \;\cup\; \Q = \R$, which is uncountable.

    This is a contradiction, and thus $\R \setminus \Q$ is uncountable.
\end{proof}

\end{document}