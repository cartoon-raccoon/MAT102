\documentclass[12pt, a4paper]{article}

\usepackage{amsmath}
\usepackage{amssymb}
\usepackage{amsthm}
\usepackage[a4paper, portrait, margin=1in]{geometry}

\newcommand{\R}{\mathbb{R}}
\newcommand{\Z}{\mathbb{Z}}
\newcommand{\N}{\mathbb{N}}
\renewcommand{\qedsymbol}{$\blacksquare$}

\newcommand{\emptyline}{\hfill\break}
\newcommand{\sse}{\subseteq}

\newtheorem{theorem}{Theorem}

% let's begin
\begin{document}

\noindent\textbf{(Q5)}

\begin{theorem}
    Let $A,\;B \sse \R$. If $f \colon A \to B$ is strictly monotone, then there
    exists a subset $C \sse \R$ with the same cardinality as $A$.
\end{theorem}

\begin{proof}
    By Proposition 5.1.4, since $f$ is strictly monotone, it is therefore injective
    on $B$.

    In order to prove that $|C| = |A|$, we construct a bijection between them.
    Let $C = f(A)$. $f$ is surjective on $C$ by the definition of function image,
    and is also injective on $C$ since $C \sse B$.

    Therefore, since $f \colon A \to C$ is bijective, $|C| = |A|$.
\end{proof}

\end{document}