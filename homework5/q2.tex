\documentclass[12pt, a4paper]{article}

\usepackage{amsmath}
\usepackage{amssymb}
\usepackage{amsthm}
\usepackage[a4paper, portrait, margin=1in]{geometry}

\newcommand{\R}{\mathbb{R}}
\newcommand{\Q}{\mathbb{Q}}
\newcommand{\Z}{\mathbb{Z}}
\newcommand{\N}{\mathbb{N}}
\renewcommand{\qedsymbol}{$\blacksquare$}

\newcommand{\emptyline}{\hfill\break}

\newtheorem{theorem}{Theorem}

% let's begin
\begin{document}

\noindent\textbf{(Q2)}

\begin{theorem}
    $\Q \setminus \Z$ is countable.
\end{theorem}

\begin{proof}
    We prove that a set is countable by defining a bijection between it and $\N$.

    Thus, we can do this by modifying the proof of countability of $\Q$ by placing
    additional constraints on how we construct the initial set.

    With reference to the proof of Theorem 5.4.1, we define for each $k \in \N$, the set

    \[
        A_k = \left\{ \frac{a}{b} \colon a, b \in \N \text{ and } 
        a + b = k \text{ and $gcd(a, b) = 1$ and } b \neq 1
        \right\}
    \]

    The reasons for adding the additional restrictions $gcd(a, b) = 1$ and $b \neq 1$
    are as follows:
    \begin{itemize}
        \item Having $a$ and $b$ be coprime prevents the rational from being reduced
        to a simpler form, and
        \item Having $b \neq 1$ prevents the fraction $\frac{a}{1}$
        from occurring, and thus resolving to an integer.
    \end{itemize}

    This also has the added benefit of removing any repetitions, e.g. $\frac{3}{6}$
    (a repetition of $\frac{1}{2}$) is not in this set since 3 and 6 are not coprime,
    as is $\frac{12}{6}$ (a repetition of $\frac{2}{1}$, which itself is not in the set).

    Thus, we end up with the following sets:

    \begin{gather*}
        A_1 = A_2 = \phi\\
        A_3 = \left\{\frac{1}{2}\right\}\\
        A_4 = \left\{\frac{1}{3}\right\}\\
        A_5 = \left\{\frac{1}{4}, \frac{2}{3}, \frac{3}{2}\right\}\\
        A_6 = \left\{\frac{1}{5}\right\}\\
        A_7 = \left\{\frac{1}{7}, \frac{2}{5}, \frac{5}{2}, \frac{3}{4}, \frac{4}{3}\right\}
    \end{gather*}

    And so on for increasing values of $k \in \N$.

    The rest of the proof follows similarly to the proof for $|\Q|$. We then arrange a sequence
    with elements from each set in order of increasing $k$:
    \[
        \underbrace{\frac{1}{2},}_{A_3}
        \underbrace{\frac{1}{3},}_{A_4}
        \underbrace{\frac{1}{4}, \frac{2}{3}, \frac{3}{2}}_{A_5}
        \underbrace{\frac{1}{5},}_{A_6}
        \underbrace{\frac{1}{7}, \frac{2}{5}, \frac{5}{2}, \frac{3}{4}, \frac{4}{3}}_{A_7}
        \ldots
    \]

    This sequence has several key properties:

    \begin{itemize}
        \item It has no repetitions, since they were all eliminated in defining $A_k$.
        \item Every non-integer positive rational appears as an element of this sequence.
        \item There are no integers in this sequence.
    \end{itemize}

    \newpage

    All that remains is to add the negative rationals, so a sequence
    $a_1, a_2, a_3, a_4, a_5$ becomes:
    \[
        a_1, -a_1, a_2, -a_2, a_3, -a_3 \ldots
    \]

    Or more explicitly:
    \[
        \frac{1}{2}, -\frac{1}{2},
        \frac{1}{3}, -\frac{1}{3}.
        \frac{1}{4}, -\frac{1}{4}, \frac{2}{3}, -\frac{2}{3},
        \frac{3}{2}, -\frac{3}{2} \ldots
    \]

    We can now construct the desired bijection $f \colon \N \to (\Q \setminus \Z)$:
    \[
        \makebox[2cm][c]{$\N$}\makebox[10cm][s]{
            1 2 3 4 5 6 7 8 9 10
        }
    \]
    \[
        \makebox[2cm][l]{}\makebox[10cm][s]{
            \mbox{$\downarrow$}
            \mbox{$\downarrow$}
            \mbox{$\downarrow$}
            \mbox{$\downarrow$}
            \mbox{$\downarrow$}
            \mbox{$\downarrow$}
            \mbox{$\downarrow$}
            \mbox{$\downarrow$}
            \mbox{$\downarrow$}
            \mbox{$\downarrow$}
        }
    \]
    \[
        \makebox[2cm][c]{$\Q \setminus \Z$}\makebox[10cm][s]{
            $\frac{1}{2}$ $-\frac{1}{2}$
            $\frac{1}{3}$ $-\frac{1}{3}$
            $\frac{1}{4}$ $-\frac{1}{4}$
            $\frac{2}{3}$ $-\frac{2}{3}$
            $\frac{3}{2}$ $-\frac{3}{2}$
        }
    \]


    This function maps odd numbers to a positive rational, and even numbers to a
    negative rational. All non-integer rationals are covered, thus the function is
    surjective; and all numbers are mapped to only once, thus the function is injective.

    Thus, we have constructed a bijection between $\N$ and $\Q \setminus \Z$, and therefore
    $\Q \setminus \Z$ is countable.

\end{proof}

\end{document}