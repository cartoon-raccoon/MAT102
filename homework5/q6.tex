\documentclass[12pt, a4paper]{article}

\usepackage{amsmath}
\usepackage{amssymb}
\usepackage{amsthm}
\usepackage[a4paper, portrait, margin=1in]{geometry}

\newcommand{\R}{\mathbb{R}}
\newcommand{\Z}{\mathbb{Z}}
\newcommand{\N}{\mathbb{N}}
\renewcommand{\qedsymbol}{$\blacksquare$}

\newcommand{\emptyline}{\hfill\break}

\newtheorem{theorem}{Theorem}

% let's begin
\begin{document}

\noindent\textbf{(Q6)}

\begin{theorem}
    Let $S$ be the set of all finite sequences of the letters \emph{a, b, c}.

    $S$ is countable.
\end{theorem}

\begin{proof}
    In order to prove that a set is countable, we form a bijection between it and $\N$.

    We thus define a bijection $f \colon S \to \N$ that maps a sequence of \textit{a, b, c}
    to a unique natural number.

    First, we form a sequence of each possible sequence of the three characters:
    \[
        a, b, c, aa, ab, ac, ba, bb, bc, \ldots, cc, aaa, aab, aac, aba, \ldots
    \]

    As soon as the character in the last place is \textit{c}, we wrap around back to 
    $a$, and "carry the $a$" to the digit on the left. In this way, we imitate a 
    ternary number system, and following this system, all possible finite and 
    unique sequences are covered.
    
    This is where we begin to form our bijection. We map each element of the above
    infinite sequence to a natural number:

    \[
        a \to 1, \; b \to 2, \; c \to 3, \; aa \to 4, \; ab \to 5, \; \ldots
    \]

    From this sequence we observe that each natural assigned to each sequence can
    be expressed as a sum of multiples of powers of 3:

    \[
        a \to 1 = 1 \cdot 3^0,\; aa \to 4 = 1 \cdot 3^1 + 1 \cdot 3^0,\; 
        ab \to 5 = 1 \cdot 3^1 + 2 \cdot 3^0, \ldots
    \]

    From this, we can form a rule for deriving a unique natural number for each element in this
    set.
    
    We first assign a digit to each number: \textit{a} to $1$, \textit{b} to $2$, and \textit{c} to $3$.

    For a sequence of $k$ length, we form its
    corresponding natural number as a sum of decreasing powers of 3, starting from $3^{k - 1}$,
    and each power multiplied by its letter's corresponding number.
    Following the example sequence $ababc$:
    \begin{center}
        \makebox[8cm][s]{$a$ $b$ $a$ $b$ $c$}

        \makebox[8cm][s]{$\downarrow$ $\downarrow$ $\downarrow$ $\downarrow$ $\downarrow$}

        \makebox[11cm][s]{
            \mbox{$1 \cdot 3^4$}
            \mbox{$2 \cdot 3^3$}
            \mbox{$1 \cdot 3^2$}
            \mbox{$2 \cdot 3^1$}
            \mbox{$3 \cdot 3^0$}
        }
    \end{center}

    Which gives us the sum $81 + 54 + 9 + 6 + 3 = 153$, i.e. $f(ababc) = 153$.

    Thus, $f \colon S \to \N$ can only produce one possible output for each unique sequence, 
    making it injective, and can produce a natural number for every sequence of $a, b, c$, making
    it surjective, thereby making it bijective.

    Since $f$ is bijective, $|\N| = |S|$, and $S$ is countable.
\end{proof}

\end{document}