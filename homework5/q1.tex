\documentclass[12pt, a4paper]{article}

\usepackage{amsmath}
\usepackage{amssymb}
\usepackage{amsthm}
\usepackage[a4paper, portrait, margin=1in]{geometry}

\newcommand{\R}{\mathbb{R}}
\newcommand{\Z}{\mathbb{Z}}
\newcommand{\N}{\mathbb{N}}
\renewcommand{\qedsymbol}{$\blacksquare$}

\newcommand{\emptyline}{\hfill\break}

\newtheorem{theorem}{Theorem}

% let's begin
\begin{document}

\noindent\textbf{(Q1)}

\begin{theorem}
    Let $f\colon A \to B$. Then there exists a subset $C \subseteq B$ such that
    $f\colon A \to C$ is bijective.
\end{theorem}

\begin{proof}
    In order to prove that $f \colon A \to C$ is bijective, we prove that it is
    both surjective and injective.
    
    Let $C = f(A)$, that is, $C$ is the image of $f$. Thus, $f$ is surjective 
    on $C$ by construction. Since $f$ is injective on $B$ and $C \subseteq B$ by 
    definition of function image, $f$ is also injective on $C$.

    Since $f \colon A \to C$ is both surjective and injective, it is therefore
    bijective.
\end{proof}

\end{document}