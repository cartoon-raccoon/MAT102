\documentclass[12pt, a4paper]{article}

\usepackage{amsmath}
\usepackage{amssymb}
\usepackage{amsthm}
\usepackage[a4paper, portrait, margin=1in]{geometry}

\newcommand{\R}{\mathbb{R}}
\newcommand{\Z}{\mathbb{Z}}
\newcommand{\N}{\mathbb{N}}
\renewcommand{\qedsymbol}{$\blacksquare$}

\newcommand{\sse}{\subseteq}

\newcommand{\emptyline}{\hfill\break}

\newtheorem{theorem}{Theorem}

% let's begin
\begin{document}

\noindent\textbf{(Q3)}

We are required to find and prove a formula for the sum $\displaystyle\sum_{i = m}^{n} i$.

This can be explicitly stated as:

\[
    m + (m + 1) + (m + 2) + (m + 3) + \ldots + (n - 1) + n
\]

We observe that this is equivalent to:

\[
    1 + 2 + \ldots + m + (m + 1) + \ldots + (n - 1) + n - (1 + 2 + \ldots + (m - 1))
\]

The summation of which can be expressed as:

\[
    \sum_{i = 1}^{n} i - \sum_{i = 1}^{m - 1} i \quad (1)
\]

We assume the following:

\[
    \sum_{i = 1}^{n} i = \frac{n(n + 1)}{2}
\]

We can thus express (1) as:

\begin{align*}
    \frac{n(n + 1)}{2} - \frac{m(m - 1)}{2} = \frac{n(n + 1) - m(m - 1)}{2}
\end{align*}

\end{document}