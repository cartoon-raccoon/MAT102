\documentclass[12pt, a4paper]{article}

\usepackage{amsmath}
\usepackage{amssymb}
\usepackage{amsthm}
\usepackage[a4paper, portrait, margin=1in]{geometry}

\newcommand{\R}{\mathbb{R}}
\newcommand{\Z}{\mathbb{Z}}
\newcommand{\N}{\mathbb{N}}
\newcommand{\st}{\text{ s.t. }}
\renewcommand{\times}{\cdot}
\renewcommand{\qedsymbol}{$\blacksquare$}

\newcommand{\sse}{\subseteq}

\newcommand{\emptyline}{\hfill\break}

\newtheorem{theorem}{Theorem}

% let's begin
\begin{document}

\noindent\textbf{(Q4)}

\noindent\textit{(a)}

$6! = 1 \cdot 2 \cdot 3 \cdot 4 \cdot 5 \cdot 6 = 720$

$7! = 6! \cdot 7 = 5040$

\noindent\textit{(b)}

\begin{proof}
    Since $n > m$, we can rewrite $n!$ as:
    \[
        (1 \times 2 \times \ldots \times m) \times ((m + 1) \times (m + 2) \times \ldots \times n)
    \]

    In product notation:

    \[
        n! = \prod_{i = 1}^{n} i = \prod_{i = 1}^{m} i \: \cdot \prod_{i = m + 1}^{n} i
    \]

    Since $n > m$ and $n, m \in \N$, $n > 1 \implies \displaystyle\prod_{i = m + 1}^{n} i > 1$.

    Therefore,

    \[
        n! = \prod_{i = 1}^{m} i \: \cdot \prod_{i = m + 1}^{n} i = m! \: \cdot \prod_{i = m + 1}^{n} i \: > m!
    \]

    Which proves the theorem, as required.
\end{proof}

\noindent\textit{(c)}

\begin{proof}
    We first consider a few base cases.

    Considering the base cases where $n = 1 \text{ and } n = 2$:
    \begin{gather*}
        x \geq 1 \implies x^1 \geq 1 \geq 1! = 1\\
        x \geq 2 \implies x^2 \geq 4 \geq 2! = 2
    \end{gather*}

    Next for our induction step, we assume the following induction hypothesis:
    \[
        x \geq n \implies x^n \geq n!
    \]

    And prove that this further implies $x^{n + 1} \geq (n + 1)!$.

    First, we observe that every time $n$ increases, $x$ must as well, in order to
    maintain $x \geq n$. Thus, let $x_2 \geq n + 1$ for some given $n$ such that $x \geq n$.

    We also observe that $(n + 1)! = n! \cdot (n + 1)$. Thus, we can rewrite the implication
    we are trying to prove as:

    \[
        x_2^{n + 1} \geq (x + 1)^{n + 1} = (x + 1)^n \cdot (x + 1) \geq n! \cdot (n + 1)
    \]

    We observe that $(x + 1)^n \geq x^n \geq n!$, and that 
    $x + 1 \geq n + 1$. Thus,
    \[
        (x + 1)^n \cdot (x + 1) \geq n! \cdot (n + 1) \implies x_2^{n + 1} \geq (n + 1)!
    \]

    Which proves the implication.

    Thus, we have proven the theorem by PMI, as required.
\end{proof}

\end{document}