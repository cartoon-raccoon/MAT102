\documentclass[12pt, a4paper]{article}

\usepackage{amsmath}
\usepackage{amssymb}
\usepackage{amsthm}
\usepackage[a4paper, portrait, margin=1in]{geometry}

\newcommand{\R}{\mathbb{R}}
\newcommand{\Z}{\mathbb{Z}}
\newcommand{\N}{\mathbb{N}}
\renewcommand{\qedsymbol}{$\blacksquare$}

\newcommand{\sse}{\subseteq}

\newcommand{\emptyline}{\hfill\break}

\newtheorem{theorem}{Theorem}

% let's begin
\begin{document}

\noindent\textbf{(Q2)}

\noindent\textit{(a)}

We observe that the series $2 + 4 + 6 + \ldots + 2n$ can be rewritten as
$2(1 + 2 + 3 + \ldots n)$. Given that $1 + 2 + 3 + \ldots + n = \frac{n(n + 1)}{2}$,
it follows:

\[
    2(1 + 2 + 3 + \ldots n) = 2 \cdot \sum_{i = 1}^{n} i = n(n + 1)
\]

\noindent\textit{(b)}

\begin{proof}
    We first consider the base case $n = 1$:
    \[
        n = 1 \implies \sum_{i = 1}^{1} 2i = 2 = 1(1 + 1)
    \]

    Thus the base case holds.

    Next, for the induction step, we assume the induction hypothesis for some natural $k$:
    \[
        \sum_{i = 1}^{k} 2i = k(k + 1)
    \]

    and prove that it implies:
    
    \[
        \sum_{i = 1}^{k + 1} 2i = \sum_{i = 1}^{k} 2i + (2k + 2) = (k + 1)((k + 1) + 1)
    \]

    It follows:
    \begin{align*}
        k(k + 1) + (2k + 2) & = k^2 + (2k + 2)\\
        & = k^2 + 3k + 2\\
        & = (k + 1)(k + 2)\\
        & = (k + 1)((k + 1) + 1)
    \end{align*}

    Thus, the induction hypothesis holds.

    Therefore, by PMI, we have proven the formula, as required.
\end{proof}

\newpage

\noindent\textit{(c)}

From (a), we observe that the series $1 + 3 + 5 + \ldots + (2n - 1)$ can be written as:
\[
    (2 - 1) + (4 - 1) + (6 - 1) + \ldots + (2n - 1)
\]

In summation notation, this is written as:
\[
    \sum_{i = 1}^{n} 2n - 1 = \sum_{i = 1}^{n} 2n - \sum_{i = 1}^{n} 1 = \left(\sum_{i = 1}^{n} 2n\right) - n
\]

We can substitute our formula in (a) to give:
\[
    n(n + 1) - n = n^2 + n - n = n^2
\]

\noindent\textit{(d)}

\begin{proof}
    We first consider our base case, where $n = 1$:
    \[
        n = 1 \implies \left(\sum_{i = 1}^{n} 2n\right) - n = 2 - 1 = 1 = n^2
    \]

    Thus, the base case holds.

    Next, in our induction step, we prove the following implication for some natural $k$:
    \[
        \left(\sum_{i = 1}^{k} 2k\right) - k = k^2 
        \implies \left(\sum_{i = 1}^{k + 1} 2i\right) - (k + 1) = (k + 1)^2
    \]

    By (a), we evaluate $\displaystyle\sum_{i = 1}^{k + 1} 2i = (k + 1)(k + 2) = k^2 + 2k + k + 2$.
    It follows:
    
    \begin{align*}
        (k^2 + 2k + k + 2) - (k + 1) & = k^2 + 2k + 1\\
        & = (k + 1)^2
    \end{align*}

    Which proves the implication.

    Thus, by PMI, we have proven the formula, as required.
\end{proof}

\end{document}