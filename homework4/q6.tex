\documentclass[12pt, a4paper]{article}

\usepackage{amsmath}
\usepackage{amssymb}
\usepackage{amsthm}
\usepackage[a4paper, portrait, margin=1in]{geometry}

\newcommand{\R}{\mathbb{R}}
\newcommand{\Z}{\mathbb{Z}}
\newcommand{\N}{\mathbb{N}}
\renewcommand{\qedsymbol}{$\blacksquare$}

\newcommand{\sse}{\subseteq}

\newcommand{\emptyline}{\hfill\break}

\newtheorem{theorem}{Theorem}

% let's begin
\begin{document}

\noindent\textbf{(Q6)}

\noindent\textit{(a)}

\begin{proof}
    We prove this with strong induction. First we consider our base case, $F_3$.

    We observe that $F_1 = 1 > 0, F_2 = 1 > 0, F_3 = 2 > 0$, so the base case holds.

    Using strong induction, we assume that for a given $k > 2$ that 
    all numbers $F_1, F_2, \ldots F_{k - 1}$ are all $> 0$.

    Now we consider our induction hypothesis, for some $k > 2$:
    \[
        F_k > 0 \implies F_{k + 1} > 0
    \]

    By the definition of the Fibonacci numbers:
    \[
        F_{k + 1} = F_k + F_{k - 1}
    \]

    We observe that by assumption, $F_k$ and $F_{k - 1} >0 $, therefore,
    $F_k + F_{k - 1} > 0 \implies F_{k + 1} > 0$, thus the induction hypothesis holds.

    Therefore, by PSMI, we have proven the theorem, as required.
\end{proof}

\noindent\textit{(b)}

\begin{proof}
    We prove this with standard induction. First we consider our base cases, $F_1, F_2$, and $F_3$:
    \[
        F_1 = 1, F_2 = 1, F_3 = 2
    \]

    Since $F_3 = 2 > F_1 = F_2$, the base case holds.

    Next, in the induction step, we assume the induction hypothesis that for some natural $k > 2$,
    $F_k > F_j$ for any $j \in \{1, 2, \ldots k - 1\}$, and prove that this implies
    that $F_{k + 1} > F_j$ for any $j \in \{1, 2, \ldots k\}$.

    By the definition of the Fibonacci numbers, we can write $F_{k + 1}$ as:
    \[
        F_{k + 1} = F_k + F_{k -1}
    \]

    We observe that by (a), $F_k > 0 \text{ and } F_{k - 1} > 0 \implies F_{k + 1} > F_k$. 
    Thus, assuming the induction hypothesis, 
    \begin{gather*}
        F_{k + 1} > F_k > F_j \text{ for any } j \in \{1, 2, \ldots k - 1\}\\
        \implies F_{k + 1} > F_j \text{ for any } j \in \{1, 2, \ldots k\}
    \end{gather*}

    Which completes the induction step.

    Since both the base case and induction hypothesis hold, by PMI we have proven the theorem,
    as required.
\end{proof}.

\end{document}