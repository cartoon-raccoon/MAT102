\documentclass[12pt, a4paper]{article}

\usepackage{amsmath}
\usepackage{amssymb}
\usepackage{amsthm}
\usepackage[a4paper, portrait, margin=1in]{geometry}

% convenience definitions for real numbers and integers.
\newcommand{\R}{\mathbb{R}}
\newcommand{\Z}{\mathbb{Z}}
\newcommand{\N}{\mathbb{N}}

\newcommand{\open}[1]{\left(#1\right)}
\newcommand{\set}[2]{\left\{#1 \:\colon\: #2\right\}}

% i like the black square better.
\renewcommand{\qedsymbol}{$\blacksquare$}

\newtheorem{theorem}{Theorem}

% let's begin
\begin{document}

\noindent\textbf{(Q7)}
\begin{theorem}
    One way to prove that $S = T$ is to prove that $S \subseteq T$ and $T \subseteq S$.

    Let
    \begin{align*}
        S & = \set{y = \R}{y = \frac{x}{x + 1}\;\text{for some}\;x \in \R \setminus \{-1\}}\\
        T & = \open{-\infty, 1} \cup \open{1, \infty} = \R \setminus \{1\}
    \end{align*}
    Use this strategy to prove that S = T.
\end{theorem}

\begin{proof}
    We observe that $S$ is the range of a function $\tfrac{x}{x + 1}$ which has the domain
    $\R \setminus \{-1\}$.
    Let $y \in S$. It follows that:
    \[
        y = \frac{x}{x + 1} \text{ for some } x \in \R \setminus \{-1\}
    \]

    Therefore, we can work our way backwards through the function to get the element from
    the domain that produced each value of $y$ in $S$.

    Thus:
    \begin{align*}
        y & = \frac{x}{x + 1}\\
        xy + y & = x\\
        y & = x - xy\\
        y & = x(1 - y)\\
        \frac{y}{1 - y} & = x
    \end{align*}  

    From the definition of $S$ and $T$,
    \[
        y \in \R \text{ and } y \neq 1 \implies y \in \R \setminus \{1\} \implies y \in T
    \]
    From which we can conclude $S \subseteq T$.

    Now, we have to prove that $T \subseteq S$.

    Similarly, we let $x \in T$. Since $1 \notin T$, we can define $T$ as the domain of
    a function that is defined everywhere except $1$.
    \begin{align*} 
        y = \frac{x}{1 - x} \text{ for some } x \in T
    \end{align*}

    It follows that:
    \begin{align*}
        y - xy & = x\\
        y & = x + xy\\
        \frac{y}{y + 1} & = x
    \end{align*}
    For all $x \in T \implies x \in S$, as $S$ is the image of the the similar function $\tfrac{x}{x + 1}$,
    thereby implying $T \subseteq S$.

    Since $S \subseteq T$ and $T \subseteq S$, we can conclude that $S = T$ by mutual
    subset inclusion, as required.
\end{proof}
\end{document}