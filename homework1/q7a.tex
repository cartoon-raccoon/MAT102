\documentclass[12pt, a4paper]{article}

\usepackage{amsmath}
\usepackage{amssymb}
\usepackage{amsthm}
\usepackage[a4paper, portrait, margin=1in]{geometry}

% convenience definitions for real numbers and integers.
\newcommand{\R}{\mathbb{R}}
\newcommand{\Z}{\mathbb{Z}}
\newcommand{\N}{\mathbb{N}}

\newcommand{\open}[1]{\left(#1\right)}
\newcommand{\set}[2]{\left\{#1 \:\colon\: #2\right\}}

% i like the black square better.
\renewcommand{\qedsymbol}{$\blacksquare$}

\newtheorem{theorem}{Theorem}

% let's begin
\begin{document}

\noindent\textbf{(Q7)}
\begin{theorem}
    One way to prove that $S = T$ is to prove that $S \subseteq T$ and $T \subseteq S$.

    Let
    \begin{align*}
        S & = \set{y = \R}{y = \frac{x}{x = 1}\;\text{for some}\;y \in \R \setminus \{-1\}}\\
        T & = \open{-\infty, 1} \cup \open{1, \infty} = \R \setminus \{1\}
    \end{align*}
    Use this strategy to prove that S = T.
\end{theorem}

\begin{proof}
    We observe that set $S$ is the image of a function $y = \tfrac{x}{x + 1}$ for the range
    of $x \in \R \setminus \{-1\}$. It thus follows that we can look at an arbitrary value
    in $S$, and work backwards through the given function to see its corresponding element
    in the range.

    Therefore, we let $a \in S$. It follows that:
    \[
    \begin{gathered}
        \begin{aligned}
            y & = \frac{x}{x + 1}\\
            \implies ax + a & = x\\
            \implies x - ax & = a\\
            \implies x(1-a) & = a\\
            \implies x & = \frac{a}{1 -a}
        \end{aligned}
    \end{gathered}
    \]
    We can see that the image $x$ of an element $a$ in set $S$ is defined by the function
    $x = \tfrac{a}{1 - a}$. In other words, $S$ can be redefined as the range of this 
    function. From this, we can see that the image of this function is undefined for 
    $a = 1$ and defined everywhere else. Thus, its range is $\R \setminus \{1\}$, 
    implying that $S \subseteq T$ where $T$ is defined as every real except $1$.

    Now, we have to prove that $T \subseteq S$.

    Similarly, we let $b \in T$. Since $1 \notin T$, we can construct $T$ as the range of
    a function that is defined everywhere except $1$. We can do this by defining the function
    as a quotient of two numbers $p$ and $1 - b$, where $p$ is an arbitrary real number:
    \begin{align*} 
        T = \{b \mid \frac{p}{1-b} \in \R \text{ for some } p \in \R\}
    \end{align*}
    This states that $b$ is an element of $T$ if the function $\tfrac{p}{1-b}$ is defined for
    that value of $b$.
    It does not matter what the value of $p$ is, as $\tfrac{p}{1-b}$ will always be undefined 
    where $b = 1$ and defined everywhere else. Since the value of $p$ does not matter, we can 
    let it be $b$.

    Thus, $T$ is the range for a function $\tfrac{b}{1 - b}$, the same as $S$ which is the range
    for a function $\tfrac{a}{1 - a}$, similarly undefined where $a = 1$. Thus, for an 
    arbitrary element $p$:
    \begin{align*}
        p \in S \text{ for all } p \in T \Rightarrow T \subseteq S
    \end{align*}

    Since $S \subseteq T$ and $T \subseteq S$, we can conclude that $S = T$ by mutual
    subset inclusion, as required.
\end{proof}
\end{document}