\documentclass[12pt, a4paper]{article}

\usepackage{amsmath}
\usepackage{amssymb}
\usepackage{amsthm}
\usepackage[a4paper, portrait, margin=1in]{geometry}

% convenience definitions for real numbers and integers.
\newcommand{\R}{\mathbb{R}}
\newcommand{\Z}{\mathbb{Z}}
\newcommand{\N}{\mathbb{N}}

\newcommand{\open}[1]{\left(#1\right)}
\newcommand{\set}[2]{\left\{#1 \:\colon\: #2\right\}}

% i like the black square better.
\renewcommand{\qedsymbol}{$\blacksquare$}

\newtheorem{theorem}{Theorem}

% let's begin
\begin{document}
\noindent\textbf{(Q5)}
\begin{theorem}
    For all sets $A,B,C$, we have $A \cup (B \cap C) = (A \cup B) \cap (A \cup C)$.
\end{theorem}

% so i don't have to keep writing this shit out
\newcommand{\A}{A \cup (B \cap C)}
\newcommand{\B}{(A \cup B) \cap (A \cup C)}
\newcommand{\sse}{\subseteq}

This theorem is true, and the proof is below.
\begin{proof}
    For the sake of clarity, we label the two sets as follows:
    \begin{align}
        & \A\\
        & \B
    \end{align}
    We can prove these two sets are equal through mutual subset inclusion. That is,
    \[
        (1) = (2) \text{ iff } (1) \sse (2) \text{ and } (2) \sse (1)
    \]

    Let $x \in \B \quad (2)$.

    This implies $x \in A$ or $B$ and $x \in A$ or $C$. From this we can see that
    $x$ is in both $B$ and $C$ or $x$ is in $A$. This can be expressed as

    \[
        x \in \A \quad (1)
    \]

    Thus, $(2) \sse (1)$.

    Now, let $x \in \A \quad (1)$.
    This can be expressed as $x \in A$ or $x \in B$ and $C$. From this, we can
    see that $x \in A$ or $x \in B$ and $x \in A$ or $x \in C$. In set notation:

    \[
        x \in \B \quad (2)
    \]

    Thus, $(1) \sse (2)$.

    Since $(1) \sse (2)$ and $(2) \sse (1)$, we can conclude $(1) = (2)$, or
    \[
        \A = \B
    \]
\end{proof}
\end{document}