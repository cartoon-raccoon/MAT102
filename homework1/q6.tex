\documentclass[12pt, a4paper]{article}

\usepackage{amsmath}
\usepackage{amssymb}
\usepackage{amsthm}
\usepackage[a4paper, portrait, margin=1in]{geometry}

% convenience definitions for real numbers and integers.
\newcommand{\R}{\mathbb{R}}
\newcommand{\Z}{\mathbb{Z}}
\newcommand{\N}{\mathbb{N}}

\newcommand{\open}[1]{\left(#1\right)}
\newcommand{\set}[2]{\left\{#1 \:\colon\: #2\right\}}

% i like the black square better.
\renewcommand{\qedsymbol}{$\blacksquare$}

\newtheorem{theorem}{Theorem}

% let's begin
\begin{document}
\noindent\textbf{(Q6)}
\begin{theorem}
    Let $S,T$ be any subsets of a universal set $U$.
    Prove that $(S \cap T)^c = S^c \cup T^c$.
\end{theorem}

\newcommand{\A}{(S \cap T)^c}
\newcommand{\B}{S^c \cup T^c}
\newcommand{\sse}{\subseteq}

\begin{proof}
    We can prove these two sets are equal by mutual subset inclusion:
    \[
        \A = \B \text { iff } \A \sse \B \text{ and } \B \sse \A
    \]

    Let $x \in \A$. This implies $x \notin (S \cap T)$, or $x \notin S$ and $T$.
    From this, we can infer $x \in S^c$ or $T^c$ which can be written as $\B$. 
    Therefore,
    \[
        x \in \B \implies \A \sse \B
    \]

    Now, let $x \in \B$. This implies $x \notin S$ or $T$. From this, we can infer
    $x \in S^c$ and $T^c$, which can be written as $\A$. Therefore,
    \[
        x \in \A \implies \B \sse \A
    \]

    Since $\A \sse \B$ and $\B \sse \A$, we can conclude $\A = \B$.
\end{proof}
\end{document}